\section{The \emph{Group-By} Operator}%
\label{gby}
We now turn our attention to the group-by operator which typically
forms the basis for aggregate function computations in SQL queries.
Common methods for implementing group-by include \textit{sorting} and
\textit{hashing} -- the specific choice of method depends both on the
constraints associated with the operator expression itself, as well as
on the downstream operators in the plan tree. We discuss below the
PCM-conscious modifications of both implementations, which share
a common number of \emph{output} tuple writes, namely $\gbcount \times \gblen$.

\subsection{Hash-Based Grouping}

A hash table entry for group-by, as compared to the corresponding
entry in hash join, has an additional field containing the aggregate
value. For each new tuple in the input array, a bucket index is
obtained after hashing the value of the column present in the group-by
expression. Subsequently, a search is made in the bucket indicated by
the index. If a tuple matching the group-by column value is found, the
aggregate field value is updated; else, a new entry is created in the
bucket. Thus, unlike hash join, where each build tuple had its individual
entry, here the grouped tuples share a common entry with an aggregate
field that is constantly updated over the course of the algorithm.

Since the hash table construction for group-by is identical to that
of the hash join operator, the PCM-related modifications described
in Section~\ref{hj} can be applied here as well. That is, we employ
a page-based hash table organization, and a reduced hash value size,
to reduce the writes to PCM.

\textbf{PCM write analysis}: From the above discussion, it is easy to see
that the total number of word-writes incurred for the PCM-conscious hash-based group-by is given by
\begin{equation}
\label{eq:gb_ht}
W_{gb\_ht} = \frac {\gbcount \times \hashsize + 
N_R \times \aggsize + \gbcount \times \gblen}{4}
\end{equation}

\subsection{Sort-Based Grouping}

Sorting may be used for group-by when a fully ordered operator such
as \textit{order by} or \textit{merge join} appears downstream in the plan
tree. Another use case is for queries with a \textit{distinct} clause
in the aggregate expression, in order to identify the duplicates that have
to be discarded from the aggregate.  

Sorting-based group-by differs in a key aspect from sorting itself
in that the sorted tuples do not have to be written out. Instead, it
is the aggregated tuples that are finally passed on to the next operator
in the plan tree. Hence, we can modify the flashsort algorithm of
Section~\ref{sort} to use \emph{pointers} in both the
Permutation and Short-range Ordering phases, subsequently leveraging these
pointers to perform aggregation on the sorted tuples. 

\textbf{PCM write analysis}: The full tuple writes of $2 N_R L_R$
which were incurred in the flashsort scheme, are
now replaced by $2N_R \times P$ since pointers are used during
both the Classification and Short-range Ordering phases. Thus, the total number of word-writes for this
algorithm for uniformly distributed data would be
\begin{equation}
\label{eq:gb_sort}
W_{gb\_sort} = \frac{2N_R \times P + \gbcount \times \gblen}{4}
\end{equation}
