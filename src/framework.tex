\section{Problem Framework}
\label{sec:framework}
In this section, we overview the problem framework, the assumptions made
in our analysis, and the notations used in the sequel.

We model the \model{} memory organization shown in Figure~\ref{fig:pcm_models} (c). 
The DRAM buffer is of size $D$, and organized in a \emph{K-way
set-associative} manner, like the L1/L2 processor cache memories. Moreover,
its operation is identical to that of an \emph{inclusive} cache in the
memory hierarchy, that is, a new DRAM line is fetched from PCM each time there
is a DRAM miss. The last level cache in turn fetches its data from the
DRAM buffer.

We assume that the writes to PCM are in word-sized units (4B) and are incurred only when a data block is evicted from DRAM to PCM. A \textit{data-comparison write (DCW)} scheme \cite{write} is used for
the writing of PCM memory blocks during eviction from DRAM -- in this
scheme, the memory controller compares the existing PCM block to the newly
evicted DRAM block, and selectively writes back only the modified words.
Further, \textit{N-Chance}~\cite{nchance} is used as the DRAM eviction
policy due to its preference for evicting non-dirty entries, thereby
saving on writes. The failure recovery mechanism for updates is orthogonal to our work and is therefore not discussed in this paper.

As described above, the simulator implements a realistic DRAM
buffer. However, in our write analyses and estimators, we assume for
tractability that there are no conflict misses in the DRAM. Thus, for
any operation dealing with data whose size is within the DRAM capacity,
our analysis assumes no evictions and consequently no writes. The experimental
evaluation in Section~\ref{sec:validation} indicates the 
impact of this assumption to be only marginal.

With regard to the operators, we use $R$ to denote the input relation
for the \textit{sort} and \textit{group-by} unary operators.  Whereas,
for the binary \textit{hash join} operator, $R$ is used to denote the
smaller relation, on which the hash table is constructed, while $S$
denotes the probing relation. 

In this paper, we assume that all input relations are \emph{completely PCM-resident}.
Further, for presentation simplicity, we assume that the sort,
hash join and group-by expressions are on singleton attributes --
the extension to multiple attributes is straightforward.

A summary of the main notation used in the analysis of the following
sections is provided in Table~\ref{tab:notations}.

\begin{table}[t]
\centering
\caption{Notations Used in Operator Analysis}
\label{tab:notations}
\begin{small}
\begin{tabular}{p{2cm}p{11cm}}
\toprule  
\textbf{Term} & \textbf{Description}\\ 
\midrule
\textbf{$D$} & DRAM size\\
\textbf{$K$} & DRAM Associativity\\
\textbf{$N_R, N_S$} & Row cardinalities of input relations R and S, respectively\\
\textbf{$L_R, L_S$} & Tuple lengths of input relations R and S, respectively\\
\textbf{$P$} & Pointer size\\
\textbf{$\hashsize$} & Size of each hash table entry\\
\textbf{$\aggsize$} & Size of aggregate field (for group-by operator)\\
\textbf{$\hjcount,\gbcount$} & Output tuple cardinalities of join and group-by operators, respectively\\
\textbf{$\hjlen,\gblen$} & Output tuple lengths of join and group-by operators, respectively\\
\bottomrule
\end{tabular}
\end{small}
\end{table}
